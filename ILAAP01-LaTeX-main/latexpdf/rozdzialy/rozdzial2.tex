\chapter{Prezentacja wybranych kreatywnych technologii}
  
  \section{Omówienie wybranych technologii}
    Współczesny świat rozwija się w niezwykle szybkim tempie, a kluczową rolę w tym procesie odgrywają technologie 
    wspierające kreatywność człowieka. Są to narzędzia, które nie tylko usprawniają pracę, lecz także poszerzają 
    możliwości twórczego wyrażania się, eksperymentowania oraz współpracy między ludźmi.  
    Nowoczesne technologie kreatywne znajdują zastosowanie w edukacji, sztuce, projektowaniu, mediach, 
    a także w biznesie i nauce.
    Rozwój takich technologii umożliwia użytkownikom tworzenie unikalnych treści wizualnych, muzycznych czy tekstowych, 
    a jednocześnie pozwala przekraczać ograniczenia tradycyjnych metod twórczych.  
    Poniższa tabela prezentuje wybrane przykłady kreatywnych technologii oraz ich znaczenie dla współczesnej kultury i gospodarki.
  \section{Tabela z przeglądem wybranych kreatywnych technologii}
    \begin{table}[H]
		\centering
		\caption{Przegląd wybranych kreatywnych technologii}
		\begin{tabular}{|p{5cm}|p{5cm}|p{5cm}|}
			\hline
			\textbf{Technologia} & \textbf{Opis} & \textbf{Przykłady zastosowania} \\ \hline
			Generatywna sztuczna inteligencja (AI) & Algorytmy zdolne do tworzenia tekstów, obrazów, muzyki i filmów na podstawie danych treningowych. & Narzędzia takie jak ChatGPT, DALL·E czy Adobe Firefly wspomagają twórców w pisaniu, projektowaniu i komponowaniu. \\ \hline
			Druk 3D & Technologia umożliwiająca fizyczne odwzorowanie trójwymiarowych projektów opracowanych cyfrowo. & Wykorzystywana w architekturze, medycynie (protezy), modzie i sztuce użytkowej. \\ \hline
      Wirtualna i rozszerzona rzeczywistość (VR/AR) & Systemy immersyjne pozwalające doświadczać świata w sposób interaktywny i wielozmysłowy. & Muzea wirtualne, szkolenia VR, interaktywne wystawy artystyczne. \\ \hline
      Narzędzia do współtworzenia online & Platformy umożliwiające współpracę nad projektami kreatywnymi w czasie rzeczywistym, niezależnie od miejsca. & Miro, Figma, Canva, Google Workspace. \\ \hline
      Blockchain w sztuce cyfrowej & Technologia rozproszonych rejestrów służąca do uwierzytelniania własności cyfrowych dzieł sztuki (NFT). & Galerie NFT, certyfikacja dzieł artystów cyfrowych. \\ \hilne
      Biotechnologia twórcza & Łączenie sztuki i nauki w celu tworzenia dzieł z użyciem organizmów żywych, tkanek lub procesów biologicznych. & BioArt – instalacje z użyciem bakterii, komórek lub DNA. \\ \hline
		\end{tabular}
	\end{table}